\documentclass[
  man,
  floatsintext,
  longtable,
  nolmodern,
  notxfonts,
  notimes,
  colorlinks=true,linkcolor=blue,citecolor=blue,urlcolor=blue]{apa7}

\usepackage{amsmath}
\usepackage{amssymb}



\usepackage[bidi=default]{babel}
\babelprovide[main,import]{english}


% get rid of language-specific shorthands (see #6817):
\let\LanguageShortHands\languageshorthands
\def\languageshorthands#1{}

\RequirePackage{longtable}
% \setlength\LTleft{0pt}
\RequirePackage{threeparttablex}

% % 


\makeatletter
\renewcommand{\paragraph}{\@startsection{paragraph}{4}{\parindent}%
	{0\baselineskip \@plus 0.2ex \@minus 0.2ex}%
	{-.5em}%
	{\normalfont\normalsize\bfseries\typesectitle}}

\renewcommand{\subparagraph}[1]{\@startsection{subparagraph}{5}{0.5em}%
	{0\baselineskip \@plus 0.2ex \@minus 0.2ex}%
	{-\z@\relax}%
	{\normalfont\normalsize\bfseries\itshape\hspace{\parindent}{#1}\textit{\addperi}}{\relax}}
\makeatother




\usepackage{longtable, booktabs, multirow, multicol, colortbl, hhline, caption, array, float}
\setcounter{topnumber}{2}
\setcounter{bottomnumber}{2}
\setcounter{totalnumber}{4}
\renewcommand{\topfraction}{0.85}
\renewcommand{\bottomfraction}{0.85}
\renewcommand{\textfraction}{0.15}
\renewcommand{\floatpagefraction}{0.7}

\usepackage{tcolorbox}
\tcbuselibrary{listings,theorems, breakable, skins}
\usepackage{fontawesome5}

\definecolor{quarto-callout-color}{HTML}{909090}
\definecolor{quarto-callout-note-color}{HTML}{0758E5}
\definecolor{quarto-callout-important-color}{HTML}{CC1914}
\definecolor{quarto-callout-warning-color}{HTML}{EB9113}
\definecolor{quarto-callout-tip-color}{HTML}{00A047}
\definecolor{quarto-callout-caution-color}{HTML}{FC5300}
\definecolor{quarto-callout-color-frame}{HTML}{ACACAC}
\definecolor{quarto-callout-note-color-frame}{HTML}{4582EC}
\definecolor{quarto-callout-important-color-frame}{HTML}{D9534F}
\definecolor{quarto-callout-warning-color-frame}{HTML}{F0AD4E}
\definecolor{quarto-callout-tip-color-frame}{HTML}{02B875}
\definecolor{quarto-callout-caution-color-frame}{HTML}{FD7E14}

\newlength\Oldarrayrulewidth
\newlength\Oldtabcolsep


\usepackage{hyperref}




\providecommand{\tightlist}{%
  \setlength{\itemsep}{0pt}\setlength{\parskip}{0pt}}
\usepackage{longtable,booktabs,array}
\usepackage{calc} % for calculating minipage widths
% Correct order of tables after \paragraph or \subparagraph
\usepackage{etoolbox}
\makeatletter
\patchcmd\longtable{\par}{\if@noskipsec\mbox{}\fi\par}{}{}
\makeatother
% Allow footnotes in longtable head/foot
\IfFileExists{footnotehyper.sty}{\usepackage{footnotehyper}}{\usepackage{footnote}}
\makesavenoteenv{longtable}

\usepackage{graphicx}
\makeatletter
\def\maxwidth{\ifdim\Gin@nat@width>\linewidth\linewidth\else\Gin@nat@width\fi}
\def\maxheight{\ifdim\Gin@nat@height>\textheight\textheight\else\Gin@nat@height\fi}
\makeatother
% Scale images if necessary, so that they will not overflow the page
% margins by default, and it is still possible to overwrite the defaults
% using explicit options in \includegraphics[width, height, ...]{}
\setkeys{Gin}{width=\maxwidth,height=\maxheight,keepaspectratio}
% Set default figure placement to htbp
\makeatletter
\def\fps@figure{htbp}
\makeatother


% definitions for citeproc citations
\NewDocumentCommand\citeproctext{}{}
\NewDocumentCommand\citeproc{mm}{%
  \begingroup\def\citeproctext{#2}\cite{#1}\endgroup}
\makeatletter
 % allow citations to break across lines
 \let\@cite@ofmt\@firstofone
 % avoid brackets around text for \cite:
 \def\@biblabel#1{}
 \def\@cite#1#2{{#1\if@tempswa , #2\fi}}
\makeatother
\newlength{\cslhangindent}
\setlength{\cslhangindent}{1.5em}
\newlength{\csllabelwidth}
\setlength{\csllabelwidth}{3em}
\newenvironment{CSLReferences}[2] % #1 hanging-indent, #2 entry-spacing
 {\begin{list}{}{%
  \setlength{\itemindent}{0pt}
  \setlength{\leftmargin}{0pt}
  \setlength{\parsep}{0pt}
  % turn on hanging indent if param 1 is 1
  \ifodd #1
   \setlength{\leftmargin}{\cslhangindent}
   \setlength{\itemindent}{-1\cslhangindent}
  \fi
  % set entry spacing
  \setlength{\itemsep}{#2\baselineskip}}}
 {\end{list}}
\usepackage{calc}
\newcommand{\CSLBlock}[1]{\hfill\break\parbox[t]{\linewidth}{\strut\ignorespaces#1\strut}}
\newcommand{\CSLLeftMargin}[1]{\parbox[t]{\csllabelwidth}{\strut#1\strut}}
\newcommand{\CSLRightInline}[1]{\parbox[t]{\linewidth - \csllabelwidth}{\strut#1\strut}}
\newcommand{\CSLIndent}[1]{\hspace{\cslhangindent}#1}


\usepackage[nolongtablepatch]{lineno}
\linenumbers



\usepackage{newtx}

\defaultfontfeatures{Scale=MatchLowercase}
\defaultfontfeatures[\rmfamily]{Ligatures=TeX,Scale=1}





\title{Too Beautiful to be Fake: Attractive Faces are Less Likely to be
Judged as Artificially Generated}
\shorttitle{Attractiveness and Reality}


\usepackage{etoolbox}








\authorsnames[{1},{2},{2},{2},{3},{1,4,5,6}]{Dominique Makowski,An Shu
Te,Stephanie Kirk,Ngoi Zi Liang,Panagiotis Mavros,S.H. Annabel Chen}







\authorsaffiliations{
{School of Psychology, University of Sussex},{School of Social
Sciences, Nanyang Technological University},{Department of Economics and
Social Sciences, Télécom Paris},{LKC Medicine, Nanyang Technological
University},{National Institute of Education},{Centre for Research and
Development in Learning, Nanyang Technological University}}






\leftheader{Makowski, Te, Kirk, Liang, Mavros and Chen}



\abstract{Technological advances render the distinction between
artificial (e.g., computer-generated faces) and real stimuli
increasingly difficult, yet the factors driving our beliefs regarding
the nature of ambiguous stimuli remain largely unknown. In this study,
150 participants rated 109 pictures of faces on 4 characteristics
(attractiveness, beauty, trustworthiness, familiarity). The stimuli were
then presented again with the new information that some of them were
AI-generated, and participants had to rate each image according to
whether they believed them to be real or fake. Despite all images being
pictures of real faces from the same database, most participants did
indeed rate a large portion of them as `fake' (often with high
confidence), with strong intra- and inter-individual variability. Our
results suggest a gender-dependent role of attractiveness on reality
judgements, with faces rated as more attractive being classified as more
real. We also report links between reality beliefs tendencies and
dispositional traits such as narcissism and paranoid ideation.}
% 
\keywords{attractiveness, AI-generated images, fiction, fake news, sense
of reality}

\authornote{\par{\addORCIDlink{Dominique
Makowski}{0000-0001-5375-9967}}\par{\addORCIDlink{An Shu
Te}{0000-0002-9312-5552}}\par{\addORCIDlink{Stephanie
Kirk}{0000-0002-9312-5552}}\par{\addORCIDlink{Panagiotis
Mavros}{0000-0002-1540-5516}}
\par{ }
\par{       Author roles were classified using the Contributor Role Taxonomy (CRediT; https://credit.niso.org/) as follows: Dominique
Makowski:   Conceptualization, Data curation, Formal Analysis, Funding
acquisition, Investigation, Methodology, Project
administration, Resources, Software, Supervision, Validation, Visualization, Writing
-- original draft; An Shu Te:   Project
administration, Resources, Writing -- original draft; Stephanie
Kirk:   Project administration, Resources, Writing -- original
draft; Ngoi Zi Liang:   Project administration, Resources, Writing --
review \& editing; Panagiotis Mavros:   Supervision, Writing -- review
\& editing; S.H. Annabel Chen:   Project
administration, Supervision, Writing -- review \& editing}
\par{Correspondence concerning this article should be addressed
to Dominique Makowski, Email: D.Makowski@sussex.ac.uk}
}


\makeatletter
\let\endoldlt\endlongtable
\def\endlongtable{
\hline
\endoldlt
}
\makeatother

\urlstyle{same}



% From https://tex.stackexchange.com/a/645996/211326
%%% apa7 doesn't want to add appendix section titles in the toc
%%% let's make it do it
\makeatletter
\xpatchcmd{\appendix}
  {\par}
  {\addcontentsline{toc}{section}{\@currentlabelname}\par}
  {}{}
\makeatother

\begin{document}

\maketitle


\setcounter{secnumdepth}{-\maxdimen} % remove section numbering

\setlength\LTleft{0pt}

\resetlinenumber[1]



Advancements in technology have now made it possible to create
near-perfect simulations that are indistinguishable from reality with an
ease, affordability and accessibility that are unprecedented in Human
history.These artificial, yet realistic constructs permeate all areas of
life through immersive works of fiction, deep fakes (real-like images
and videos generated by deep learning algorithms), virtual and augmented
reality (VR and AR), artificial beings (artificial intelligence ``bots''
with or without a physical form), fake news and skewed narratives, of
which ground truth is often hard to access (Nightingale \& Farid, 2022).
Such developments not only carry important consequences for the
technological and entertainment sectors, but also for security and
politics - for instance if used for propaganda and disinformation,
recruitment into malevolent organizations, or religious indoctrination
(Pantserev, 2020). This issue is central to what has been coined the
``post-truth era'' (Lewandowsky et al., 2017), in which the distinction
(and lack thereof) between authentic and simulated objects will play a
critical role.

While not all simulations have achieved perfect realism, such as
Computer Generated Images (CGI) in movies or via recent algorithms such
as GANs or diffusion model, which often include distortions or lack
certain key details distinguishing them from real images (Corvi et al.,
2022; McDonnell \& Breidt, 2010), it is fair to assume that these
technical limitations will become negligible in the near future. This is
particularly true in the field of face generation, where face-generation
algorithms are already able to create stimuli that are virtually
indistinguishable from real photos (Moshel et al., 2022; Nightingale \&
Farid, 2022; Tucciarelli et al., 2020). Such a technological feat,
however, leads to a new question: if real and fake stimuli cannot be
differentiated based on their objective ``physical'' characteristics,
how can we form judgements regarding their nature?

Literature shows that the context surrounding a stimulfus often plays an
important role in the assessment of its reality (a process henceforth
referred to as \emph{simulation monitoring}, Makowski, 2018; Makowski,
Sperduti, et al., 2019). With the extensive search and processing of
cues within ambiguous stimuli being an increasingly complex and
cognitively effortful strategy (Michael \& Sanson, 2021; Susmann et al.,
2021), people tend to draw on peripheral contextual cues (\textbf{Figure
1}), such as the source of the stimulus (e.g., which journal was the
information published in), and its credibility, authority and expertise,
to help facilitate their evaluation (Michael \& Sanson, 2021; Petty \&
Cacioppo, 1986; Susmann et al., 2021). However, the atomization and
decontextualization of information allowed by online social media (where
text snippets or video excerpts are often mass-shared with little
context) makes this task progressively difficult (Berghel, 2018; Y. Chen
et al., 2015). Thus, in the absence of clear contextual information,
what drives our beliefs of reality?

\begin{figure}[H]

\caption{Figure 1. The decision to believe that an ambiguous stimulus
(of any form, e.g., images, text, videos, environments, \ldots) is real
or fake depends of individual characteristics (e.g., personality and
cognitive styles), stimulus-related features (context, emotionality),
and their interaction, which can manifest for instance in our bodily
reaction.}

{\centering \includegraphics[width=1\textwidth,height=\textheight]{../../figures/figure1.png}

}

\end{figure}%

Evidence suggests that inter-individual characteristics play a crucial
role in simulation monitoring, with factors such as cognitive style,
prior beliefs, and personality traits (Bryanov \& Vziatysheva, 2021;
Ecker et al., 2022; Sindermann et al., 2020). For instance, individuals
with stronger analytical reasoning skills have been found to better
discriminate real from fake stimuli (Pehlivanoglu et al., 2021;
Pennycook \& Rand, 2019), and prior knowledge or beliefs about the
stimulus influences one's perception of it by biasing the attention
deployment towards information that is in line with one's expectations
(Britt et al., 2019). Furthermore, dispositional traits, such as high
levels of narcissism and low levels of openness and conscientiousness,
have been associated with greater susceptibility to fake news (Piksa et
al., 2022; Sindermann et al., 2020). \textbf{Interestingly, a recent
review suggested that narcissism was related to a strong self-perceived
ability at detecting lies (Turi et al., 2022), which could translate to
participants scoring high on narcissism providing more clear cut and
confidence responses. Conversely, those high in honesty-humility tend to
be more conservative in their judgments to ensure fairness (Liu et al.,
2020), likely resulting in lower confidence ratings.}

Beyond stimulus- and individual-related characteristics, evidence
suggests that the interaction between the two (i.e., the subjective
reaction associated with the experience of a given stimulus),
contributes to simulation monitoring decisions. For instance, the
intensity of experienced emotions have been shown to increase one's
sense of presence - the extent to which one feels like ``being there'',
as if the object of experience was real - when engaged in a fictional
movie or a VR environment (Makowski et al., 2017; Sanchez-Vives \&
Slater, 2005). Indeed, participants' self-reported emotional arousal
were found to significantly predict the probability that they would
perceive images as real (Azevedo et al., 2020). Conversely, beliefs that
emotional stimuli were fake (e.g., that emotional scenes were not
authentic but instead involved actors and movie makeup) were found to
result in emotion down-regulation (Makowski, Sperduti, et al., 2019;
Sperduti et al., 2017). In line with these findings, studies on
susceptibility to fake news have also found heightened stimulus
emotionality to be associated with greater belief (Bago et al., 2022;
Martel et al., 2020), and higher neurophysiological arousal was
predictive of judging realistic images as real (Azevedo et al., 2020).
Additionally, other factors, such as the stimuli's perceived
self-relevance (Goldstein, 2009; Sperduti et al., 2016), as well as
familiarity (Begg et al., 1992), could also play a role in guiding our
appraisal of a stimulus.\textbf{For instance,} Miller et al. (2023)
\textbf{reported that participants were more likely to mistakenly
identify AI-generated faces as real because they perceived them as more
familiar.}

Due to their popularity as a target of CGI technology and the prospect
offered with facial features that can be experimentally manipulated,
AI-generated images of faces are increasingly used to study face
processing (Dawel et al., 2021), in particular in relationship with
saliency or emotions, as well as to other important components of face
evaluation, such as trustworthiness or attractiveness (Balas \& Pacella,
2017; Calbi et al., 2017; Sobieraj \& Krämer, 2014; Tsikandilakis et
al., 2019). Interestingly, artificially created faces rated as more
attractive (by an independent group of raters) were perceived as less
real (Tucciarelli et al., 2020). Conversely, Liefooghe et al. (2022)
reports that attractiveness ratings were significantly lower when
participants who were told that the faces were AI-generated were
compared to those who had no prior knowledge. \textbf{Similarly, when
participants are informed that faces are AI-generated, the perceived
artificiality leads to lower trust ratings (Wang \& Nishida, 2024), even
when they are real faces (Liefooghe et al., 2022). In contrast, when
participants are unaware that the faces are AI-generated, trust ratings
for these synthetic faces tend to increase (Nightingale \& Farid,
2022)}. Whereas this line of evidence suggests that reality beliefs have
an effect on face attractiveness and \textbf{trustworthiness} ratings,
the opposite question - whether attractiveness \textbf{and
trustworthiness contribute} to the formation of reality beliefs - has
received little attention to date.

AI-generated content, in particular realistic images, is becoming
commonplace and carries important risks for misinformation and
black-mailing (Viola \& Voto, 2023), emphasizing the need to understand
the different components that come into play in the formation of reality
beliefs. This exploratory study primarily aims at investigating the
effect of facial attractiveness on simulation monitoring, i.e., on the
beliefs that an image is real or artificially generated. Based on the
\emph{affective reality theory} (Makowski, 2018, 2023), which suggests
that salient and emotional stimuli are perceived to be more real (up to
a point of reversal after which beliefs of fiction becomes used an
emotion regulation strategy), we hypothesize a quadratic relationship
between perceived realness and attractiveness: faces rated as highly
attractive or unattractive will more likely be believed to be real. We
expect a similar relationship with trustworthiness ratings given its
well-established link with attractiveness (Bartosik et al., 2021;
Garrido \& Prada, 2017; Liefooghe et al., 2022; Little et al., 2011),
and a positive relationship with familiarity (as more familiar faces
would appear as more salient, self-relevant and anchored in reality).
Additionally, we will further explore the link shared by dispositional
traits, such as personality and attitude towards AI, with simulation
monitoring tendencies. Importantly, this study does investigate the
discriminative accuracy between ``true'' photos and ``true''
artificially-generated images (which we consider more a technological
issue than a psychological one), focusing on the beliefs that a stimulus
is real or fake, independently of its true. \textbf{In other words, the
present study investigates the psychological process that leads to
different \emph{beliefs} of reality, rather than the discrimination
between real faces and actual AI-generated ones, which largely depends
on the technological quality of the AI-generation process.}

\subsection{Methods}\label{methods}

All the material (preregistration \textbf{\emph{{[}FOOTNOTE: This
approach diverges from the preregistration in several key ways. First,
the phrasing of items was modified from ``Assuming the face you saw was
of a real individual, how\ldots{}'' to ``I find this person\ldots{}'' A
new attractiveness scale (i.e., Beauty) was introduced to capture a more
objective measure of attraction. Finally, the data analysis method was
altered from Bayesian Mixed Models due to computational limitations, as
we were unable to run these models on a high-performance cluster.{]}}},
experiment demo, experiment code, raw data, analysis script with
complementary figures and analyses, etc.) is available at
\href{https://github.com/RealityBending/FakeFace}{\textbf{https://github.com/RealityBending/FakeFace}}.

\textbf{Ethics Statement.} This study was approved by the NTU
Institutional Review Board (NTU IRB-2022-187) and all procedures
performed were in accordance with the ethical standards of the
institutional board and with the 1964 Helsinki Declaration. All
participants provided their informed consent prior to participation and
were incentivized after completing the study.

\textbf{Procedure.} In the first part of the study, participants
answered a series of personality questionnaires presented in the order
below. These include the \emph{Mini-IPIP6} (24 items, Sibley et al.,
2011) measuring 6 personality traits, the \emph{SIAS-6} and the
\emph{SPS-6} (6 items each, Peters et al., 2012) assessing social
anxiety levels, 5 items we devised pertaining to expectations about
AI-generated image technology (``I think current Artificial Intelligence
algorithms can generate very realistic images''), \textbf{to potentially
test and mitigate the potential effect of expectations/beliefs about
AI.} \textbf{These items were} mixed with 5 items from the general
attitudes towards AI scale to lower the former's saliency and the
possibility of it priming the subjects about the task, (\emph{GAAIS},
Schepman \& Rodway, 2020) the \emph{FFNI-BF} (30 items, Jauk et al.,
2022) measuring 9 facets of narcissism; the \emph{R-GPTS} (18 items,
Freeman et al., 2021) measuring 2 dimensions related to paranoid
thinking; and the \emph{IUS-12} (12 items, Carleton et al., 2007)
measuring intolerance to uncertainty. Self-rated attractiveness was also
assessed using 2 items - one measuring general attractiveness (``How
attractive would you say you are?'' Marcinkowska et al., 2021) and the
other measuring physical attractiveness (``How would you rate your own
physical attractiveness relative to the average,'' Spielmann et al.,
2020). 3 attention check questions were also embedded in the surveys.
\textbf{All Cronbach's alpha values were within the acceptable to
excellent range, except for the neuroticism subscale of the Mini-IPIP6
and the negative subscale of the GAAI, which were poor, and the
Expectations about AI scale, which was questionable (Gliem \& Gliem,
2003; see supplementary material for the details of the reliability
analysis).}

In the second part of this study, images of neutral-expression faces
from the validated American Multiracial Face Database (AMFD, J. M. Chen
et al., 2021) were presented to the participants for 500ms each, in a
randomized order, following a fixation cross display (750 ms).
T\textbf{he decision to present the faces for 500 ms was based on pilot
studies, which demonstrated that this duration provides a sufficient
perceptual window for decision-making and aligns with previous research
indicating stable judgment levels and increased confidence beyond this
exposure time (Willis \& Todorov, 2006).}

\textbf{The AMFD is a recently validated database including a set of 110
pictures of homogeneous quality featuring diverse faces (particularly in
terms of ethnicity), each (except one) posing with either a neutral or
smiling expression. We selected all 109 neutral images (89 women and 20
men) to reduce the influence of confounding factors like affect. The
AMFD primarily features racially ambiguous faces, representing multiple
racial categories such as multiracial, Latinx, and white. The database
includes 81 faces from individuals self-reporting two racial backgrounds
and 29 from those with three or more racial backgrounds: 33\%
Asian/White, 22\% Latinx/White, 11\% Asian/Latinx, 6\% White/Middle
Eastern, 5\% Black/White, and 5\% Asian/Middle Eastern, with about 18\%
identifying as other racial backgrounds.}

After each stimulus presentation, ratings of \emph{Trustworthiness} (``I
find this person trustworthy'') and \emph{Familiarity} (``This person
reminds me of someone I know'') were collected using visual analog
scales. Notably, as facial attractiveness is a multidimensional
construct, encompassing evolutionary, sociocultural, biological as well
as cognitive aspects (Han et al., 2018; Rhodes et al., 2006), we
assessed attractiveness using 2 visual analog scales, measuring general
\emph{Attractiveness} (``I find this person attractive'') and physical
\emph{Beauty} (``This face is good-looking''). \textbf{This dual-scale
approach aims to reflect two conceptually distinct dimensions:
Attractiveness might capture personal, Self-relevant and subjective
appeal, whereas Beauty might be related to a more ``objective'' decision
based on aesthetic criteria that can be recognized independently of
personal attraction. In other words, we wanted the experiment to be able
to potentially capture scenarios where a face could be judged beautiful
yet not, attractive and vice versa.}

In the last part of the study, participants were informed that ``about
half'' of the images previously seen were AI-generated (the instructions
used a cover story explaining that the aim of the research was to
validate a new face generation algorithm). The same set of stimuli was
displayed again for 500 ms in a new randomized order. This time, after
each display, participants were asked to express their belief regarding
the nature of the stimulus using a visual analog scale (with \emph{Fake}
and \emph{Real} as the two extremes). The study was implemented using
\emph{jsPsych} (De Leeuw, 2015), and the exact instructions are
available in the experiment code.

\textbf{Participants.} Although the main part of the study relied on
within-subject design (with 109 trials per participant), we also planned
to do between-participants analyses, thus aiming at collecting a larger
sample than traditionally used in experimental psychology (with budget
availability as the main constraint). One hundred and fifty participants
were recruited via \emph{Prolific}, a crowd-sourcing platform recognized
for providing high quality data {[}Peer et al. (2022);
douglas2023data{]}. The only inclusion criterion was a fluent
proficiency in English to ensure that the experiment instructions would
be well-understood. Participants were incentivised with a reward of
about \textsterling 7.5 for completing the study, which took about 45
minutes to finish. Demographic variables (age, gender, sexual
orientation, education and ethnicity) were self-reported on a voluntary
basis.

We excluded 5 participants that either failed 2 (\textgreater= 66.6\%)
or more attention check questions, took an implausibly short time to
finish the questionnaires or had incomplete responses. Out of the 5
participants excluded, 2 Participants were excluded because they failed
2 out of 3 attention checks, 1 because they did not answer the sexual
orientation question, which made further analysis impossible, and 2 had
an abnormal low agreement (r \textless{} 0.1) between the beauty and the
attractiveness ratings (possibly indicating random responses as these
two scales exhibited a higher correlation for the other
participants).The final sample included 145 participants (Mean age =
28.3, SD = 9.0, range: {[}19, 66{]}; Sex:48.3\% females, 51.0\% males,
0.7\% others).

\textbf{Data Analysis.} The real-fake ratings (measured originally with
a {[}-1, 1{]} analog scale) were converted into two scores,
corresponding to two conceptually distinct mechanisms: the dichotomous
\emph{belief} (real or fake, based on the sign of the rating) and the
\emph{confidence} (the rating's absolute value) associated with that
belief. The former was analyzed using logistic mixed models, which
modelled the probability of assigning a face to the real (\textgreater=
0) as opposed to fake (\textless{} 0). The latter, as well as the other
face ratings (attractiveness, beauty, trustworthiness and familiarity),
was modelled using mixed beta regressions (suited for outcome variables
expressed in percentages). The models included the participants and
stimuli as random \textbf{intercepts with no nested variables}.

We started by investigating the effect of the procedure and instructions
to check whether the stimuli (which were all images of real faces) were
judged as fake in sufficient proportion to warrant their analysis.
Additionally, we assessed the effect of the re-exposure delay, i.e., the
time between the first presentation of the image (corresponding to the
face ratings) and the second presentation (for the real-fake rating), as
well as that of the presentation order to check whether for habituation
or learning effects.

The determinants of reality beliefs were modelled separately for
attractiveness, beauty, trustworthiness, and familiarity, using second
order raw polynomials coefficients to allow for possible quadratic
relationships (\textbf{Figure 2}). Aside from attractiveness
(conceptualized as a general construct), models for beauty,
trustworthiness and familiarity were adjusted for the the two remaining
variables \emph{mutatis mutandis}. The analysis focused on
sexual-orientation relevant stimuli, i.e., on faces that were aligned
with respect to the participants' sexual orientation (i.e., female faces
for heterosexual males, male faces for homosexual males, etc.), and the
models included the interaction with the participants' gender (as a
sexual dimorphism has been reported in face appraisal processes). For
the attractiveness and beauty models, we then added the interaction with
the reported self-attractiveness (the average of the two questions
pertaining to it) to investigate its potential modulatory effect.
Finally, we investigated the inter-individual correlates of simulation
monitoring with similar models (but this time, for all items regardless
of the participant's gender or sexual orientation) for each
questionnaire, with all of the subscales as orthogonal predictors.\\

\begin{figure}[H]

\caption{Figure 2. Top part shows the efffect of face ratings on 1) the
probability of judging a face as real vs.~fake (solid line) and 2) on
the confidence associated with that judgement (dashed lines) depending
on the sex. Bottom part shows the effect of personality traits on the
belief (black line) and the confidence associated with it (colored
lines). The points are the average per participant confidence for both
types of judgements. Stars indicate significance (p \textless{}
.001\emph{\textbf{, p \textless{} .01}, p \textless{} .05}).}

{\centering \includegraphics[width=1\textwidth,height=\textheight]{../../figures/Figure2.png}

}

\end{figure}%

The analysis was carried out using \emph{R 4.2} (R Core Team, 2022), the
\emph{tidyverse} (Wickham et al., 2019), and the \emph{easystats}
collection of packages (Lüdecke et al., 2019, 2020, 2021; Makowski,
Ben-Shachar, et al., 2019; Makowski et al., 2020). As all the details,
scripts and complimentary analyses are open-access, we will focus in the
manuscript on findings that are highly statistically significant
(\(p <.01\)).

\subsection{Results}\label{results}

On average, across participants, 44\% of images (95\%\textasciitilde CI
{[}0.12, 0.64{]}) were judged as fake and 56\% of images
(95\%\textasciitilde CI {[}0.36, 0.88{]}) as real. An intercept-only
model with the participants and images as random factors showed that the
Intraclass Correlation Coefficient (ICC), which can be interpreted as
the proportion of variance explained by the random factors, was of 9.0\%
for the participants and 9.6\% for the stimuli.

While the delay of stimulus re-exposure stimulus did not have a
significant effect on participants' beliefs of reality (\(OR = 1.00\),
\(95\%~CI = [0.99, 1.00]\)), judgement confidence was found to be
negatively associated with re-exposure delay when the faces were judged
as real (\(\beta = -0.006\), \(95\%~CI = [-0.1, 0.002]\), \(p = .004\)).
The presentation order also did not have have an effect on the belief
(\(OR = 1.00\), \(95\%~CI = [1.00, 1.00]\)) but was related to a
decrease of confidence (\(\beta_{real} = -0.003\),
\(95\%~CI = [-0.004, -0.002]\), \(p < .001\); \(\beta_{fake} = -0.002\),
\(95\%~CI = [-0.004, -0.0003]\), \(p = .021\)): items presented at the
end of the session were judged with a similar bias but a decreased
overall confidence.

\textbf{Determinants of Simulation Monitoring.} Attractiveness had a
significant positive and linear relationship (\(R^2_{marginal}\) =
2.0\%) with the belief that a stimulus was real
(\(\beta_{poly1} = 16.57\), \(95\%~CI = [7.33, 25.82]\), \(z = 3.51\),
\(p < .001\)) for males, and a quadratic relationship for females
(\(\beta_{poly2} = 7.82\), \(95\%~CI = [1.81, 13.84]\), \(z = 2.55\),
\(p = .011\)), with both non-attractive and attractive faces being
judged as more real. Attractiveness was also found to have a significant
positive and quadratic relationship with confidence in judging faces
both as real (\(\beta_{poly2} = 4.30\), \(95\%~CI = [0.97, 7.64]\),
\(z = 2.53\), \(p = .011\)) and as fake (\(\beta_{poly2} = 5.23\),
\(95\%~CI = [0. 86, 9.60]\), \(z = 2.35\), \(p = .019\)) for females.
For males, however, a significant negative and quadratic relationship
was found between attractiveness ratings and belief confidence only for
faces judged as fake (\(\beta_{poly2} = -9.92\),
\(95\%~CI = [-18.99, -0.86]\), \(z = -2.15\), \(p = .032\)). There was
no interaction with reported self-attractiveness.

Beauty, adjusted for trustworthiness and familiarity, had a significant
positive and linear relationship (\(R^2_{marginal}\) = 2.0\%) with the
belief that a stimulus was real (\(\beta_{poly1} = 11.82\),
\(95\%~CI = [4.28, 20.21]\), \(z = 2.76\), \(p = .006\)) for males only.
No effect on confidence was found, aside from a quadratic relationship
in females for faces judged as fake, suggesting that non-beautiful and
highly beautiful faces were rated as fake with more confidence than
average faces (\(\beta_{poly2} = 7.84\), \(95\%~CI = [3.39, 12.29]\),
\(z = 3.46\), \(p < .001\)). There was no interaction with reported
self-attractiveness.

Trustworthiness, adjusted for beauty and familiarity, had a
predominantly positive and linear relationship (\(R^2_{marginal}\) =
2.0\%) with the belief that a stimulus was real
(\(\beta_{poly1} = 6.44\), \(95\%~CI = [-0.11, 13.00]\), \(z = 1.93\),
\(p = .0054\)) for females only. No effect on confidence was found for
males, whereas a quadratic relationship was found for females for both
faces judged as real (\(\beta_{poly2} = 6.14\),
\(95\%~CI = [2.13, 10.14]\), \(z = 3.00\), \(p = .003\)) as well as fake
(\(\beta_{poly2} =6.12\), \(95\%~CI = [1.49, 10.75]\), \(z = 2.59\),
\(p = .001\)), suggesting that non-trustworthy and highly trustworthy
faces were rated with more confidence than average faces.

We did not find any significant relationships for familiarity adjusted
for beauty and trustworthiness (\(R^2_{marginal}\) = 2.0\%). However, a
significant positive and linear relationship was found between
familiarity and the confidence judgements of rating faces as real
(\(\beta_{poly1} = 9.98\), \(95\%~CI = [3.83, 16.13]\), \(z = 3.18\),
\(p = .001\)) whereas a negative linear relationship was found with
those judged as fake (\(\beta_{poly1} = -12.41\),
\(95\%~CI = [-20.27, -4.54]\), \(z = -3.09\), \(p = .002\)) for males
only. This hence suggests that males more confidently judge faces as
real with when they are familiar, and as fake when they are unfamiliar.

Note that we also tested as predictors the normative attractiveness and
trustworthiness scores (i.e., the average values from the stimuli
database validation), which showed a significant positive linear
relationship between beliefs of reality and attractiveness, as well as
trustworthiness, only for males (see Supplementary Analysis for
details).

\textbf{Inter-Individual Correlates of Simulation Monitoring.} The
models including the personality traits suggested that
\emph{Honesty-Humility} had a significant negative relationship with the
confidence associated with real as well as fake judgements
(\(\beta_{real} = -1.62\), \(95\%~CI = [-2.55, -0.70]\), \(z = -3.43\),
\(p < .001\); \(\beta_{fake} = -1.16\), \(95\%~CI = [-2.09, -0.23]\),
\(z = -2.45\), \(p = 0.014\)).

Significant positive associations were found between the probability of
judging faces as real and dimensions of narcissism such as \emph{Acclaim
Seeking} (\(\beta = 2.24\), \(95\%~CI = [1.17, 4.27]\), \(z = 2.44\),
\(p = .015\)), and \emph{Manipulativeness} (\(\beta = 0.47\),
\(95\%~CI = [0.25, 0.87]\), \(z = -2.4\), \(p = 0.017\)). Confidence
judgements also shared significant links with narcissism through various
facets, such as a positive relationship between the confidence for both
real and fake judgements with \emph{Acclaim Seeking}
(\(\beta_{real} = 1.65\), \(95\%~CI = [0.59, 2.70]\), \(z = 3.07\),
\(p =.002\); \(\beta_{fake} = 1.62\), \(95\%~CI = [0.56, 2.68]\),
\(z = 3.00\), \(p = .003\)), and a negative relationship with
\emph{Authoritativeness} (\(\beta_{real} = -1.57\),
\(95\%~CI = [-2.58, -0.57]\), \(z = -3.08\), \(p = .002\);
\(\beta_{fake} = -1.49\), \(95\%~CI = [-2.50, -0.48]\), \(z = -2.89\),
\(p = .004\)).

A positive trend was found in the relationship between the
\emph{Persecutory Ideation} dimension of paranoid thinking and the
belief that the faces were real (\(\beta = 1.87\),
\(95\%~CI = [0.99, 3.54]\), \(z = 1.93\), \(p = .054\)).

The \emph{Prospective Anxiety} aspect of intolerance to uncertainty
shared a negative trend in its association with confidence ratings
(\(\beta_{real} = 1.43\), \(95\%~CI = [0.10, 2.76]\), \(z = 2.10\),
\(p = .036\); \(\beta_{fake} = -0.91\), \(95\%~CI = [-1.93, 0.11]\),
\(z = -1.75\), \(p = .081\)). No significant effect was found for social
anxiety.

Questions pertaining to the attitude towards AI were reduced to 3
dimensions through factor analysis, labelled AI-Enthusiasm (loaded by
items expressing interest and excitement in AI development and
applications), AI-Realness (loaded by items expressing positive opinions
on the ability of AI to create realistic material), and AI-Danger
(loaded by items expressing concerns on the unethical misuse of AI
technology). However, only AI-Enthusiasm displayed a significant
positive relationship with the confidence in both real and fake
judgements (\(\beta_{real} = 0.21\), \(95\%~CI = [0.02, 0.40]\),
\(z = 2.20\), \(p = .028\); \(\beta_{fake} = 0.31\),
\(95\%~CI = [0.12, 0.50]\), \(z = -8.90\), \(p < 0.001\)).

\subsection{Discussion}\label{discussion}

This study aimed at investigating the effect of facial ratings
(attractiveness, beauty, trustworthiness and familiarity) on simulation
monitoring, i.e., on the belief that a stimulus was artificially
generated. Most strikingly, despite all the stimuli being real faces
from the same database, all participants believed (to high degrees of
confidence) that a significant proportion of them were fake. This
finding not only attests to the effectiveness of our instructions, but
highlights the current levels of expectation regarding CGI technology.
The strong impact of prior expectations and information on reality
beliefs underlines the volatility of our sense of reality. In fact,
stimuli-related and participant-related characteristics accounted
together for less than 20\% of the beliefs variance, suggesting a large
contribution of other subjective processes.

Although attractiveness did not seem to be the primary drive underlying
simulation monitoring of face images, we do nonetheless report
significant associations, with different patterns observed depending on
the participant's gender. The quadratic relationship found for female
participants is aligned with our hypothesis that salient faces (i.e.,
rated as very attractive or very unattractive) are judged to be more
real. The fact that this effect did not reach significance for beauty
underlines that attractiveness judgement, and its role in simulation
monitoring, is a multidimensional construct that cannot be reduced to
physical facial attractiveness, in particular for women (Buunk et al.,
2002; Qi \& Ying, 2022). In fact, female participants were more
confident in judging faces as fake only when they were rated very high
or low on beauty, suggesting that physical beauty and attractiveness are
not analogous in their effects on simulation monitoring decisions.

Interestingly, we found a significant positive linear relationship in
male participants for both attractiveness and beauty on simulation
monitoring that we could interpret under an evolutionary lens.
Specifically, males purportedly place more emphasis on facial
attractiveness as a sign of reproductive potential, as compared with
females, who tend to value characteristics signaling resource
acquisition capabilities (Buunk et al., 2002; Fink et al., 2006; Qi \&
Ying, 2022). It is thus possible that the evolutionary weight associated
with attractiveness skewed the perceived saliency of attractive faces
for men, rendering them significantly more salient than unattractive
faces, and in turn distorting the relationship with simulation
monitoring. However, future studies should test this saliency-based
hypothesis by measuring constructs closer to salience and its effects,
for instance using neuroimaging (Indovina \& Macaluso, 2007; Lou et al.,
2015) or physiological markers (e.g., heart rate deceleration, Skora et
al., 2022).

Our findings do not support the existence of a strong link between
perceived trustworthiness and reality judgments. Given prior evidence
that faces presented as computer-generated were rated less trustworthy
(Balas \& Pacella, 2017; Hoogers, 2021; Liefooghe et al., 2022), we
expected such a linear association to be more clearly present. However,
our results suggest a relationship with confidence ratings, especially
for women, whereby faces judged with low and high trustworthiness are
judged as real and fake with higher confidence. One of the underlying
mechanisms that possibly contributed to this dimorphism could be the
increased risk-taking aversion reported in females (explained
evolutionarily as a compromise to their reproductive potential, Van Den
Akker et al., 2020), to which perceived facial trustworthiness relates
(Hou \& Liu, 2019). Future studies should clarify the role of
trustworthiness both as a predictor and outcome of reality beliefs.

Contrary to our hypothesis, we did not find familiarity to be
significantly related to simulation monitoring decisions. Interestingly,
there were significant linear relationships between familiarity and
confidence judgements for males only, where familiarity increased the
confidence of reality beliefs. Although the familiarity measure was not
a ``recognition'' measure, evidence from studies pertaining to the
latter could be linked, reporting better face memory for females (Lewin
\& Herlitz, 2002; Mishra et al., 2019; Sommer et al., 2013), as well as
an overconfidence in face recall for males (Bailey, 2021; Herbst, 2020).
However, it should be noted that the distribution of familiarity ratings
was strongly skewed, and only a low number of pictures was rated as
highly familiar. As such, future studies should clarify this point by
experimentally manipulating familiarity, for instance by modulating the
amount of exposure to items before querying the simulation monitoring
judgements.

Regarding the role of inter-individual characteristics in simulation
monitoring tendencies, we found higher scores of honesty-humility - a
trait related to an increased risk perception and aversion (Levidi et
al., 2022; Weller \& Thulin, 2012) - to be related to a lower confidence
in simulation monitoring judgements. Notably, greater narcissistic
tendencies in dimensions such as acclaim seeking were associated with a
higher number of faces judged as real. This is in line with recent
research which found people with higher narcissism scores less likely to
engage in analytical reasoning strategies such as reflective thinking
(Ahadzadeh et al., 2021; Littrell et al., 2020), and to be more vigilant
and attentive to external stimuli (Carolan, 2017; Eddy, 2021; Grapsas et
al., 2020).

Moreover, putting the significant positive links between narcissistic
acclaim seeking and confidence judgements in perspective with the
negative correlation between honesty-humility and narcissism (Hodson et
al., 2018), we confirm previous evidence regarding the relationship
between narcissistic grandiosity and over-confidence in decision-making
(Brunell \& Buelow, 2017; Campbell et al., 2004; Chatterjee \& Pollock,
2017; O'Reilly \& Hall, 2021). Although an inverse effect was found for
the narcissistic facet of authoritativeness, we interpret this
relationship as related to a higher response assertiveness. Taken
together, these results suggest that participants with low humility and
high recognition desires are more confident in their judgement regarding
the real or fake nature of ambiguous stimuli. Alternatively,
participants with opposite traits might perceive a higher risk in the
decision-making process and its potential consequences (e.g., being seen
as bad at the task at hand), resulting in more conservative confidence
ratings.

Our findings suggest - though with low certainty - a potential positive
link between paranoid ideation and the tendency to believe that the
stimuli were real. Given previous reports that people with higher levels
of paranoia are more sensitive to cues of social threat
(Fornells-Ambrojo et al., 2015; Freeman et al., 2003; King \& Dudley,
2017), it is plausible that paranoid traits confer greater saliency and
emotionality to observed faces, hence increasing perceptions of its
realness. This hypothesis, if confirmed by future studies, would be in
line with previous findings that persecutory delusions are predicted by
a greater sense of presence in VR environments populated with virtual
characters (Freeman et al., 2005).

Despite the ubiquity of AI, the literature pertaining to the influence
of people's AI attitudes on simulation monitoring is scarce. Contrary to
our expectations, we did not find evidence for the role of participants'
expectations regarding the capabilities of AI technology (in terms of
the realism of its productions). Instead, we found only one's enthusiasm
about AI technology to be related to an increased confidence in
simulation monitoring ratings. This could potentially be because
participants with a highly positive attitude towards AI perceive
themselves as having greater knowledge about AI and its capabilities
(Said et al., 2022), hence permitting themselves to be more confident in
their simulation monitoring decisions. In fact, this result is in line
with reports that AI attitudes interacts with people's perceived
self-knowledge to influence their perception of the opportunities and
risks accorded by AI applications (Said et al., 2022).

On a methodological level, although the order of presentation of the
facial images was randomized to reduce effects of adaptation,
participants were more confident in their judgements for faces perceived
as real following a shorter re-exposure delay. Such shorter durations
could be associated with the faces being better remembered and appearing
more familiar, thereby triggering self-referential and autobiographical
memory processing during the repeated display (Abraham \& Von Cramon,
2009; Gobbini et al., 2013; Taylor et al., 2009). Indeed, this finding
is consistent with studies in which fictional stimuli that were
associated with familiarity up-regulated emotions, biasing its salience
and perceived realness (Makowski et al., 2017; Sperduti et al., 2016).
However, if that was the case, we would expect shorter re-exposure
delays to impact the decision bias as well towards reality, rather than
simply the confidence. Future studies should further investigate the
modulatory effects of types and degrees of familiarity on perceived
realness judgements.

Several limitations have to be noted. The current experimental paradigm
required participants to judge the realness of faces they had prior
exposure to (which was done to prevent reality judgements from
influencing the other ratings). Although the effect of re-exposure delay
was negligible, the potential bias induced by face familiarity, that is
by re-presenting the same face stimuli twice, as compared to judging
completely new items, cannot be discarded. Future studies could examine
that by incorporating novel face images or increasing the duration of
the re-exposure delay.

Another issue is the impact on reality judgements of the prior explicit
instruction that ``about half of the faces were AI-generated and the
other half real photos''. Given this prior information given to
participants, it might seems like our enthusiasm pertaining to the
finding that most people did indeed believe a high number of stimuli to
be fake might be unwarranted, since it simply affirms participants
followed the instructions. However, even if that was the case, the
finding that our beliefs of reality can be so easily re-programmed with
simple instructions and lead to high-confidence answers remains an
interesting phenomenon. Moreover, it is to note that the paradigm did
not instruct participants to balance their answers according to a
certain distribution (e.g., 50-50), merely providing them a description
of the dataset. The fact that no presentation order effect was found on
reality beliefs suggests that participants did not try to actively
distribute their responses to match the instructions, in which case we
would have expected a different pattern: for instance, the first few
items judged as real (the initial ``true'' belief of the participants),
and a bias would progressively appear towards responding ``fake'' (as
participants realize that all stimuli are of similar nature and that
they have to ``make up'' for the prevalence of their ``real'' answers to
fulfil the expected proportion of responses given the instructions).

That said, the potential demand effect of the instructions still exists,
and a control condition without the cover story with AI-generated images
would in-principle be able to mitigate such confounds to some extent.
However, the distinction real/fake is hard to operationalize and
introduce to participants in a vacuum (simply instructing them to
discriminate real from fake without providing some background
information regarding the context and defining what is meant by ``fake''
seems hardly feasible). That being the case, future studies should study
the impact of these higher-order expectations on ratings (for instance,
Tucciarelli et al., 2020 found that merely mentioning that some faces
were AI generated decreased, on average, the trustworthiness ratings for
all faces) as well as on the simulation monitoring process itself (i.e.,
the ``criterion'': would people form and distribute judgements
differently). This can be studied by modulating this expectation in a
controlled fashion (e.g., ``most of the images \emph{but a few} are
real'' vs.~``most of the images \emph{but a few} are fake'') or
inventing some implicit way of measuring reality belief that would not
require the explicit introduction of the concept of fake vs.~real to
participants.

Finally, it is important to note that although consistent in their
directions across models and variables, the magnitude of the effects
found in the study was relatively small, suggesting that the facial
appraisals measured in the study were not the key determinants of
simulation monitoring. Hence, beyond exploring new potential mechanisms,
future studies should include a more thorough debriefing to try to
capture what conscious strategies (if any) the participants used (e.g.,
focusing on some features of the stimulus - like hair or eyes in the
case of faces) to guide their reality beliefs. \textbf{Additionally, the
role of specific facial features, like perceived dominance, warmth or
gender, would be an interesting avenue to explore in future studies, in
particular with paradigms directly manipulating these dimensions (for
instance using AI to generate faces of different characteristics).}

In summary, the aim of the present study was to examine whether a subset
of specific characteristics, in particular face attractiveness,
significantly influences our simulation monitoring decisions. Notably,
we found faces rated as attractive to be perceived as more real, with a
possible sexual dimorphism affecting the shape of the relationship. We
also found that inter-individual traits, such as narcissistic
acclaim-seeking, honesty-humility, and paranoid ideation, were related
to a systematic bias towards beliefs that the stimuli were real or fake.
We believe that these findings provide the foundations to help us
understand what drives reality beliefs in an increasingly
reality-ambiguous world.

\section{Data Availability}\label{data-availability}

The datasets generated and/or analysed during the current study are
available in the GitHub repository
https://github.com/RealityBending/FakeFace

\pagebreak

\section{References}\label{references}

\phantomsection\label{refs}
\begin{CSLReferences}{1}{0}
\bibitem[\citeproctext]{ref-abraham2009reality}
Abraham, A., \& Von Cramon, D. Y. (2009). Reality= relevance? Insights
from spontaneous modulations of the brain's default network when telling
apart reality from fiction. \emph{PloS One}, \emph{4}(3), e4741.

\bibitem[\citeproctext]{ref-ahadzadeh2021social}
Ahadzadeh, A. S., Ong, F. S., \& Wu, S. L. (2021). Social media
skepticism and belief in conspiracy theories about COVID-19: The
moderating role of the dark triad. \emph{Current Psychology}, 1--13.

\bibitem[\citeproctext]{ref-azevedo2020body}
Azevedo, R., Tucciarelli, R., De Beuklaer, S., Ambroziak, K., Jones, I.,
\& Tsakiris, M. (2020). \emph{A body of evidence:`feeling in
seeing'predicts realness judgments for photojournalistic images.}

\bibitem[\citeproctext]{ref-bago2022emotion}
Bago, B., Rosenzweig, L. R., Berinsky, A. J., \& Rand, D. G. (2022).
Emotion may predict susceptibility to fake news but emotion regulation
does not seem to help. \emph{Cognition and Emotion}, 1--15.

\bibitem[\citeproctext]{ref-bailey2021gender}
Bailey, A. (2021). \emph{A gender in-group effect on facial recall}
{[}PhD thesis{]}. University of Tasmania.

\bibitem[\citeproctext]{ref-balas2017}
Balas, B., \& Pacella, J. (2017). Trustworthiness perception is
disrupted in artificial faces. \emph{Computers in Human Behavior},
\emph{77}. \url{https://doi.org/10.1016/j.chb.2017.08.045}

\bibitem[\citeproctext]{ref-bartosik2021you}
Bartosik, B., Wojcik, G. M., Brzezicka, A., \& Kawiak, A. (2021). Are
you able to trust me? Analysis of the relationships between personality
traits and the assessment of attractiveness and trust. \emph{Frontiers
in Human Neuroscience}, \emph{15}, 685530.

\bibitem[\citeproctext]{ref-begg1992dissociation}
Begg, I. M., Anas, A., \& Farinacci, S. (1992). Dissociation of
processes in belief: Source recollection, statement familiarity, and the
illusion of truth. \emph{Journal of Experimental Psychology: General},
\emph{121}(4), 446.

\bibitem[\citeproctext]{ref-berghel2018weaponizing}
Berghel, H. (2018). Weaponizing twitter litter: Abuse-forming networks
and social media. \emph{Computer}, \emph{51}(4), 70--73.

\bibitem[\citeproctext]{ref-britt2019reasoned}
Britt, M. A., Rouet, J.-F., Blaum, D., \& Millis, K. (2019). A reasoned
approach to dealing with fake news. \emph{Policy Insights from the
Behavioral and Brain Sciences}, \emph{6}(1), 94--101.

\bibitem[\citeproctext]{ref-brunell2017narcissism}
Brunell, A. B., \& Buelow, M. T. (2017). Narcissism and performance on
behavioral decision-making tasks. \emph{Journal of Behavioral Decision
Making}, \emph{30}(1), 3--14.

\bibitem[\citeproctext]{ref-bryanov2021determinants}
Bryanov, K., \& Vziatysheva, V. (2021). Determinants of individuals'
belief in fake news: A scoping review determinants of belief in fake
news. \emph{PLoS One}, \emph{16}(6), e0253717.

\bibitem[\citeproctext]{ref-buunk2002age}
Buunk, B. P., Dijkstra, P., Fetchenhauer, D., \& Kenrick, D. T. (2002).
Age and gender differences in mate selection criteria for various
involvement levels. \emph{Personal Relationships}, \emph{9}(3),
271--278.

\bibitem[\citeproctext]{ref-calbi2017}
Calbi, M., Heimann, K., Barratt, D., Siri, F., Umiltà, M. A., \&
Gallese, V. (2017). How context influences our perception of emotional
faces: A behavioral study on the kuleshov effect. \emph{Frontiers in
Psychology}, \emph{8}.
\url{https://www.frontiersin.org/articles/10.3389/fpsyg.2017.01684}

\bibitem[\citeproctext]{ref-campbell2004narcissism}
Campbell, W. K., Goodie, A. S., \& Foster, J. D. (2004). Narcissism,
confidence, and risk attitude. \emph{Journal of Behavioral Decision
Making}, \emph{17}(4), 297--311.

\bibitem[\citeproctext]{ref-carleton2007fearing}
Carleton, R. N., Norton, M. P. J., \& Asmundson, G. J. (2007). Fearing
the unknown: A short version of the intolerance of uncertainty scale.
\emph{Journal of Anxiety Disorders}, \emph{21}(1), 105--117.

\bibitem[\citeproctext]{ref-carolan2017searching}
Carolan, P. L. (2017). \emph{Searching {``inaffectively''}: A
behavioral, psychometric, and electroencephalographic investigation of
psychopathic personality and visual-spatial attention} {[}PhD thesis{]}.
Arts \& Social Sciences: Department of Psychology.

\bibitem[\citeproctext]{ref-chatterjee2017master}
Chatterjee, A., \& Pollock, T. G. (2017). Master of puppets: How
narcissistic CEOs construct their professional worlds. \emph{Academy of
Management Review}, \emph{42}(4), 703--725.

\bibitem[\citeproctext]{ref-chen2021broadening}
Chen, J. M., Norman, J. B., \& Nam, Y. (2021). Broadening the stimulus
set: Introducing the american multiracial faces database. \emph{Behavior
Research Methods}, \emph{53}(1), 371--389.

\bibitem[\citeproctext]{ref-chen2015news}
Chen, Y., Conroy, N. K., \& Rubin, V. L. (2015). News in an online
world: The need for an {``automatic crap detector.''} \emph{Proceedings
of the Association for Information Science and Technology},
\emph{52}(1), 1--4.

\bibitem[\citeproctext]{ref-corvi2022detection}
Corvi, R., Cozzolino, D., Zingarini, G., Poggi, G., Nagano, K., \&
Verdoliva, L. (2022). On the detection of synthetic images generated by
diffusion models. \emph{arXiv Preprint arXiv:2211.00680}.

\bibitem[\citeproctext]{ref-dawel2021systematic}
Dawel, A., Miller, E. J., Horsburgh, A., \& Ford, P. (2021). A
systematic survey of face stimuli used in psychological research
2000--2020. \emph{Behavior Research Methods}, 1--13.

\bibitem[\citeproctext]{ref-de2015jspsych}
De Leeuw, J. R. (2015). jsPsych: A JavaScript library for creating
behavioral experiments in a web browser. \emph{Behavior Research
Methods}, \emph{47}(1), 1--12.

\bibitem[\citeproctext]{ref-ecker2022psychological}
Ecker, U. K., Lewandowsky, S., Cook, J., Schmid, P., Fazio, L. K.,
Brashier, N., Kendeou, P., Vraga, E. K., \& Amazeen, M. A. (2022). The
psychological drivers of misinformation belief and its resistance to
correction. \emph{Nature Reviews Psychology}, \emph{1}(1), 13--29.

\bibitem[\citeproctext]{ref-eddy2021self}
Eddy, C. M. (2021). Self-serving social strategies: A systematic review
of social cognition in narcissism. \emph{Current Psychology}, 1--19.

\bibitem[\citeproctext]{ref-fink2006facial}
Fink, B., Neave, N., Manning, J. T., \& Grammer, K. (2006). Facial
symmetry and judgements of attractiveness, health and personality.
\emph{Personality and Individual Differences}, \emph{41}(3), 491--499.

\bibitem[\citeproctext]{ref-fornells2015people}
Fornells-Ambrojo, M., Freeman, D., Slater, M., Swapp, D., Antley, A., \&
Barker, C. (2015). How do people with persecutory delusions evaluate
threat in a controlled social environment? A qualitative study using
virtual reality. \emph{Behavioural and Cognitive Psychotherapy},
\emph{43}(1), 89--107.

\bibitem[\citeproctext]{ref-freeman2005psychology}
Freeman, D., Garety, P. A., Bebbington, P., Slater, M., Kuipers, E.,
Fowler, D., Green, C., Jordan, J., Ray, K., \& Dunn, G. (2005). The
psychology of persecutory ideation II: A virtual reality experimental
study. \emph{The Journal of Nervous and Mental Disease}, \emph{193}(5),
309--315.

\bibitem[\citeproctext]{ref-freeman2021revised}
Freeman, D., Loe, B. S., Kingdon, D., Startup, H., Molodynski, A.,
Rosebrock, L., Brown, P., Sheaves, B., Waite, F., \& Bird, J. C. (2021).
The revised green et al., paranoid thoughts scale (r-GPTS): Psychometric
properties, severity ranges, and clinical cut-offs. \emph{Psychological
Medicine}, \emph{51}(2), 244--253.

\bibitem[\citeproctext]{ref-freeman2003can}
Freeman, D., Slater, M., Bebbington, P. E., Garety, P. A., Kuipers, E.,
Fowler, D., Met, A., Read, C. M., Jordan, J., \& Vinayagamoorthy, V.
(2003). Can virtual reality be used to investigate persecutory ideation?
\emph{The Journal of Nervous and Mental Disease}, \emph{191}(8),
509--514.

\bibitem[\citeproctext]{ref-garrido2017kdef}
Garrido, M. V., \& Prada, M. (2017). KDEF-PT: Valence, emotional
intensity, familiarity and attractiveness ratings of angry, neutral, and
happy faces. \emph{Frontiers in Psychology}, \emph{8}, 2181.

\bibitem[\citeproctext]{ref-gobbini2013prioritized}
Gobbini, M. I., Gors, J. D., Halchenko, Y. O., Rogers, C., Guntupalli,
J. S., Hughes, H., \& Cipolli, C. (2013). Prioritized detection of
personally familiar faces. \emph{PloS One}, \emph{8}(6), e66620.

\bibitem[\citeproctext]{ref-goldstein2009pleasure}
Goldstein, T. R. (2009). The pleasure of unadulterated sadness:
Experiencing sorrow in fiction, nonfiction, and" in person.".
\emph{Psychology of Aesthetics, Creativity, and the Arts}, \emph{3}(4),
232.

\bibitem[\citeproctext]{ref-grapsas2020and}
Grapsas, S., Brummelman, E., Back, M. D., \& Denissen, J. J. (2020). The
{``why''} and {``how''} of narcissism: A process model of narcissistic
status pursuit. \emph{Perspectives on Psychological Science},
\emph{15}(1), 150--172.

\bibitem[\citeproctext]{ref-han2018beauty}
Han, S., Li, Y., Liu, S., Xu, Q., Tan, Q., \& Zhang, L. (2018). Beauty
is in the eye of the beholder: The halo effect and generalization effect
in the facial attractiveness evaluation. \emph{Acta Psychologica
Sinica}, \emph{50}(4), 363.

\bibitem[\citeproctext]{ref-herbst2020gender}
Herbst, T. H. (2020). Gender differences in self-perception accuracy:
The confidence gap and women leaders' underrepresentation in academia.
\emph{SA Journal of Industrial Psychology}, \emph{46}(1), 1--8.

\bibitem[\citeproctext]{ref-hodson2018dark}
Hodson, G., Book, A., Visser, B. A., Volk, A. A., Ashton, M. C., \& Lee,
K. (2018). Is the dark triad common factor distinct from low
honesty-humility? \emph{Journal of Research in Personality}, \emph{73},
123--129.

\bibitem[\citeproctext]{ref-hoogers2021effect}
Hoogers, E. (2021). \emph{The effect of attitude towards computer
generated faces on face perception} {[}\{B.S.\} thesis{]}.

\bibitem[\citeproctext]{ref-hou2019survival}
Hou, C., \& Liu, Z. (2019). The survival processing advantage of face:
The memorization of the (un) trustworthy face contributes more to
survival adaptation. \emph{Evolutionary Psychology}, \emph{17}(2),
1474704919839726.

\bibitem[\citeproctext]{ref-indovina2007dissociation}
Indovina, I., \& Macaluso, E. (2007). Dissociation of stimulus relevance
and saliency factors during shifts of visuospatial attention.
\emph{Cerebral Cortex}, \emph{17}(7), 1701--1711.

\bibitem[\citeproctext]{ref-jauk2022validation}
Jauk, E., Olaru, G., Schürch, E., Back, M. D., \& Morf, C. C. (2022).
Validation of the german five-factor narcissism inventory and
construction of a brief form using ant colony optimization.
\emph{Assessment}, 10731911221075761.

\bibitem[\citeproctext]{ref-king2017paranoia}
King, A., \& Dudley, R. (2017). Paranoia, worry, cognitive avoidance and
intolerance of uncertainty in a student population. \emph{Journal of
Applied Psychology and Social Science}, \emph{3}(2), 70--89.

\bibitem[\citeproctext]{ref-levidi2022understanding}
Levidi, M. D. C., McGrath, A., Kyriakoulis, P., \& Sulikowski, D.
(2022). Understanding criminal decision-making: Links between
honesty-humility, perceived risk and negative affect: Psychology, crime
\& law. \emph{Psychology, Crime and Law}, 1--29.

\bibitem[\citeproctext]{ref-lewandowsky2017beyond}
Lewandowsky, S., Ecker, U. K., \& Cook, J. (2017). Beyond
misinformation: Understanding and coping with the {``post-truth''} era.
\emph{Journal of Applied Research in Memory and Cognition}, \emph{6}(4),
353--369.

\bibitem[\citeproctext]{ref-lewin2002sex}
Lewin, C., \& Herlitz, A. (2002). Sex differences in face
recognition---women's faces make the difference. \emph{Brain and
Cognition}, \emph{50}(1), 121--128.

\bibitem[\citeproctext]{ref-liefooghe2022faces}
Liefooghe, B., Oliveira, M., Leisten, L. M., Hoogers, E., Aarts, H., \&
Hortensius, R. (2022). \emph{Faces merely labelled as artificial are
trusted less}.

\bibitem[\citeproctext]{ref-little2011facial}
Little, A. C., Jones, B. C., \& DeBruine, L. M. (2011). Facial
attractiveness: Evolutionary based research. \emph{Philosophical
Transactions of the Royal Society B: Biological Sciences},
\emph{366}(1571), 1638--1659.

\bibitem[\citeproctext]{ref-littrell2020overconfidently}
Littrell, S., Fugelsang, J., \& Risko, E. F. (2020). Overconfidently
underthinking: Narcissism negatively predicts cognitive reflection.
\emph{Thinking \& Reasoning}, \emph{26}(3), 352--380.

\bibitem[\citeproctext]{ref-liu2020honesty}
Liu, J., Zettler, I., \& Hilbig, B. E. (2020). Honesty-humility.
\emph{Encyclopedia of Personality and Individual Differences},
1996--2004.

\bibitem[\citeproctext]{ref-lou2015perceptual}
Lou, B., Hsu, W.-Y., \& Sajda, P. (2015). Perceptual salience and reward
both influence feedback-related neural activity arising from choice.
\emph{Journal of Neuroscience}, \emph{35}(38), 13064--13075.

\bibitem[\citeproctext]{ref-parametersArticle}
Lüdecke, D., Ben-Shachar, M., Patil, I., \& Makowski, D. (2020).
Extracting, computing and exploring the parameters of statistical models
using {R}. \emph{Journal of Open Source Software}, \emph{5}(53), 2445.
\url{https://doi.org/10.21105/joss.02445}

\bibitem[\citeproctext]{ref-performanceArticle}
Lüdecke, D., Ben-Shachar, M., Patil, I., Waggoner, P., \& Makowski, D.
(2021). {performance}: An {R} package for assessment, comparison and
testing of statistical models. \emph{Journal of Open Source Software},
\emph{6}(60), 3139. \url{https://doi.org/10.21105/joss.03139}

\bibitem[\citeproctext]{ref-insightArticle}
Lüdecke, D., Waggoner, P., \& Makowski, D. (2019). Insight: A unified
interface to access information from model objects in {R}. \emph{Journal
of Open Source Software}, \emph{4}(38), 1412.
\url{https://doi.org/10.21105/joss.01412}

\bibitem[\citeproctext]{ref-makowski2018cognitive}
Makowski, D. (2018). \emph{Cognitive neuropsychology of implicit emotion
regulation through fictional reappraisal} {[}PhD thesis{]}. Sorbonne
Paris Cit{é}.

\bibitem[\citeproctext]{ref-makowski2023affective}
Makowski, D. (2023). How do we know what is real? The 'affective reality
theory'. In \emph{Dr Dominique Makowski}.
\url{https://dominiquemakowski.github.io/post/2023-04-11-affectivereality/}

\bibitem[\citeproctext]{ref-bayestestRArticle}
Makowski, D., Ben-Shachar, M., \& Lüdecke, D. (2019). {bayestestR}:
Describing effects and their uncertainty, existence and significance
within the {Bayesian} framework. \emph{Journal of Open Source Software},
\emph{4}(40), 1541. \url{https://doi.org/10.21105/joss.01541}

\bibitem[\citeproctext]{ref-correlationArticle}
Makowski, D., Ben-Shachar, M., Patil, I., \& Lüdecke, D. (2020). Methods
and algorithms for correlation analysis in {R}. \emph{Journal of Open
Source Software}, \emph{5}(51), 2306.
\url{https://doi.org/10.21105/joss.02306}

\bibitem[\citeproctext]{ref-makowski2017being}
Makowski, D., Sperduti, M., Nicolas, S., \& Piolino, P. (2017). {``Being
there''} and remembering it: Presence improves memory encoding.
\emph{Consciousness and Cognition}, \emph{53}, 194--202.

\bibitem[\citeproctext]{ref-makowski2019phenomenal}
Makowski, D., Sperduti, M., Pelletier, J., Blondé, P., La Corte, V.,
Arcangeli, M., Zalla, T., Lemaire, S., Dokic, J., Nicolas, S., et al.
(2019). Phenomenal, bodily and brain correlates of fictional reappraisal
as an implicit emotion regulation strategy. \emph{Cognitive, Affective,
\& Behavioral Neuroscience}, \emph{19}(4), 877--897.

\bibitem[\citeproctext]{ref-marcinkowska2021self}
Marcinkowska, U. M., Jones, B. C., \& Lee, A. J. (2021). Self-rated
attractiveness predicts preferences for sexually dimorphic facial
characteristics in a culturally diverse sample. \emph{Scientific
Reports}, \emph{11}(1), 1--8.

\bibitem[\citeproctext]{ref-martel2020reliance}
Martel, C., Pennycook, G., \& Rand, D. G. (2020). Reliance on emotion
promotes belief in fake news. \emph{Cognitive Research: Principles and
Implications}, \emph{5}(1), 1--20.

\bibitem[\citeproctext]{ref-mcdonnell2010face}
McDonnell, R., \& Breidt, M. (2010). Face reality: Investigating the
uncanny valley for virtual faces. In \emph{ACM SIGGRAPH ASIA 2010
sketches} (pp. 1--2).

\bibitem[\citeproctext]{ref-michael2021source}
Michael, R. B., \& Sanson, M. (2021). Source information affects
interpretations of the news across multiple age groups in the united
states. \emph{Societies}, \emph{11}(4), 119.

\bibitem[\citeproctext]{ref-miller2023}
Miller, E. J., Steward, B. A., Witkower, Z., Sutherland, C. A.,
Krumhuber, E. G., \& Dawel, A. (2023). AI hyperrealism: Why AI faces are
perceived as more real than human ones. \emph{Psychological Science},
\emph{34}(12), 1390--1403.

\bibitem[\citeproctext]{ref-mishra2019gender}
Mishra, M. V., Likitlersuang, J., B Wilmer, J., Cohan, S., Germine, L.,
\& DeGutis, J. M. (2019). Gender differences in familiar face
recognition and the influence of sociocultural gender inequality.
\emph{Scientific Reports}, \emph{9}(1), 1--12.

\bibitem[\citeproctext]{ref-moshel2022}
Moshel, M. L., Robinson, A. K., Carlson, T. A., \& Grootswagers, T.
(2022). Are you for real? Decoding realistic AI-generated faces from
neural activity. \emph{Vision Research}, \emph{199}, 108079.
\url{https://doi.org/10.1016/j.visres.2022.108079}

\bibitem[\citeproctext]{ref-nightingale2022}
Nightingale, S. J., \& Farid, H. (2022). AI-synthesized faces are
indistinguishable from real faces and more trustworthy.
\emph{Proceedings of the National Academy of Sciences}, \emph{119}(8),
e2120481119. \url{https://doi.org/10.1073/pnas.2120481119}

\bibitem[\citeproctext]{ref-o2021grandiose}
O'Reilly, C. A., \& Hall, N. (2021). Grandiose narcissists and decision
making: Impulsive, overconfident, and skeptical of experts--but seldom
in doubt. \emph{Personality and Individual Differences}, \emph{168},
110280.

\bibitem[\citeproctext]{ref-pantserev2020}
Pantserev, K. (2020). \emph{The malicious use of AI-based deepfake
technology as the new threat to psychological security and political
stability} (pp. 37--55).
\url{https://doi.org/10.1007/978-3-030-35746-7_3}

\bibitem[\citeproctext]{ref-peer2022}
Peer, E., Rothschild, D., Gordon, A., Evernden, Z., \& Damer, E. (2022).
Data quality of platforms and panels for online behavioral research.
\emph{Behavior Research Methods}, \emph{54}(4), 1643--1662.
\url{https://doi.org/10.3758/s13428-021-01694-3}

\bibitem[\citeproctext]{ref-pehlivanoglu2021role}
Pehlivanoglu, D., Lin, T., Deceus, F., Heemskerk, A., Ebner, N. C., \&
Cahill, B. S. (2021). The role of analytical reasoning and source
credibility on the evaluation of real and fake full-length news
articles. \emph{Cognitive Research: Principles and Implications},
\emph{6}(1), 1--12.

\bibitem[\citeproctext]{ref-pennycook2019lazy}
Pennycook, G., \& Rand, D. G. (2019). Lazy, not biased: Susceptibility
to partisan fake news is better explained by lack of reasoning than by
motivated reasoning. \emph{Cognition}, \emph{188}, 39--50.

\bibitem[\citeproctext]{ref-peters2012development}
Peters, L., Sunderland, M., Andrews, G., Rapee, R. M., \& Mattick, R. P.
(2012). Development of a short form social interaction anxiety (SIAS)
and social phobia scale (SPS) using nonparametric item response theory:
The SIAS-6 and the SPS-6. \emph{Psychological Assessment}, \emph{24}(1),
66.

\bibitem[\citeproctext]{ref-petty1986elaboration}
Petty, R. E., \& Cacioppo, J. T. (1986). The elaboration likelihood
model of persuasion. In \emph{Communication and persuasion} (pp. 1--24).
Springer.

\bibitem[\citeproctext]{ref-piksa2022cognitive}
Piksa, M., Noworyta, K., Piasecki, J., Gwiazdzinski, P., Gundersen, A.
B., Kunst, J., \& Rygula, R. (2022). Cognitive processes and personality
traits underlying four phenotypes of susceptibility to (mis)
information. \emph{Frontiers in Psychiatry}, 1142.

\bibitem[\citeproctext]{ref-qi2022gender}
Qi, Y., \& Ying, J. (2022). Gender biases in the accuracy of facial
judgments: Facial attractiveness and perceived socioeconomic status.
\emph{Frontiers in Psychology}, \emph{13}.

\bibitem[\citeproctext]{ref-RCoreTeam2022}
R Core Team. (2022). \emph{R: A language and environment for statistical
computing}. R Foundation for Statistical Computing.
\url{https://www.R-project.org/}

\bibitem[\citeproctext]{ref-rhodes2006evolutionary}
Rhodes, G. et al. (2006). The evolutionary psychology of facial beauty.
\emph{Annual Review of Psychology}, \emph{57}, 199.

\bibitem[\citeproctext]{ref-said2022artificial}
Said, N., Potinteu, A.-E., Brich, I., Buder, J., Schumm, H., \& Huff, M.
(2022). \emph{An artificial intelligence perspective: How knowledge and
confidence shape risk and opportunity perception}.

\bibitem[\citeproctext]{ref-sanchez2005presence}
Sanchez-Vives, M. V., \& Slater, M. (2005). From presence to
consciousness through virtual reality. \emph{Nature Reviews
Neuroscience}, \emph{6}(4), 332--339.

\bibitem[\citeproctext]{ref-schepman2020initial}
Schepman, A., \& Rodway, P. (2020). Initial validation of the general
attitudes towards artificial intelligence scale. \emph{Computers in
Human Behavior Reports}, \emph{1}, 100014.

\bibitem[\citeproctext]{ref-sibley2011}
Sibley, C., Luyten, N., Wolfman, M., Mobberley, A., Wootton, L. W.,
Hammond, M., Sengupta, N., Perry, R., West-Newman, T., Wilson, M.,
McLellan, L., Hoverd, W. J., \& Robertson, A. (2011). The mini-IPIP6:
Validation and extension of a short measure of the big-six factors of
personality in new zealand. \emph{New Zealand Journal of Psychology},
\emph{40}, 142--159.

\bibitem[\citeproctext]{ref-sindermann2020short}
Sindermann, C., Cooper, A., \& Montag, C. (2020). A short review on
susceptibility to falling for fake political news. \emph{Current Opinion
in Psychology}, \emph{36}, 44--48.

\bibitem[\citeproctext]{ref-skora2022functional}
Skora, L., Livermore, J., \& Roelofs, K. (2022). The functional role of
cardiac activity in perception and action. \emph{Neuroscience \&
Biobehavioral Reviews}, 104655.

\bibitem[\citeproctext]{ref-sobieraj2014beautiful}
Sobieraj, S., \& Krämer, N. C. (2014). What is beautiful in cyberspace?
Communication with attractive avatars. \emph{International Conference on
Social Computing and Social Media}, 125--136.

\bibitem[\citeproctext]{ref-sommer2013sex}
Sommer, W., Hildebrandt, A., Kunina-Habenicht, O., Schacht, A., \&
Wilhelm, O. (2013). Sex differences in face cognition. \emph{Acta
Psychologica}, \emph{142}(1), 62--73.

\bibitem[\citeproctext]{ref-sperduti2016paradox}
Sperduti, M., Arcangeli, M., Makowski, D., Wantzen, P., Zalla, T.,
Lemaire, S., Dokic, J., Pelletier, J., \& Piolino, P. (2016). The
paradox of fiction: Emotional response toward fiction and the modulatory
role of self-relevance. \emph{Acta Psychologica}, \emph{165}, 53--59.

\bibitem[\citeproctext]{ref-sperduti2017distinctive}
Sperduti, M., Makowski, D., Arcangeli, M., Wantzen, P., Zalla, T.,
Lemaire, S., Dokic, J., Pelletier, J., \& Piolino, P. (2017). The
distinctive role of executive functions in implicit emotion regulation.
\emph{Acta Psychologica}, \emph{173}, 13--20.

\bibitem[\citeproctext]{ref-spielmann2020predictive}
Spielmann, S. S., Maxwell, J. A., MacDonald, G., Peragine, D., \&
Impett, E. A. (2020). The predictive effects of fear of being single on
physical attractiveness and less selective partner selection strategies.
\emph{Journal of Social and Personal Relationships}, \emph{37}(1),
100--123.

\bibitem[\citeproctext]{ref-susmann2021persuasion}
Susmann, M. W., Xu, M., Clark, J. K., Wallace, L. E., Blankenship, K.
L., Philipp-Muller, A. Z., Luttrell, A., Wegener, D. T., \& Petty, R. E.
(2021). Persuasion amidst a pandemic: Insights from the elaboration
likelihood model. \emph{European Review of Social Psychology}, 1--37.

\bibitem[\citeproctext]{ref-taylor2009neural}
Taylor, M. J., Arsalidou, M., Bayless, S. J., Morris, D., Evans, J. W.,
\& Barbeau, E. J. (2009). Neural correlates of personally familiar
faces: Parents, partner and own faces. \emph{Human Brain Mapping},
\emph{30}(7), 2008--2020.

\bibitem[\citeproctext]{ref-tsikandilakis2019beauty}
Tsikandilakis, M., Bali, P., \& Chapman, P. (2019). Beauty is in the eye
of the beholder: The appraisal of facial attractiveness and its relation
to conscious awareness. \emph{Perception}, \emph{48}(1), 72--92.

\bibitem[\citeproctext]{ref-tucciarelli2020}
Tucciarelli, R., Vehar, N., \& Tsakiris, M. (2020). \emph{On the
realness of people who do not exist: the social processing of artificial
faces}. \url{https://doi.org/10.31234/osf.io/dnk9x}

\bibitem[\citeproctext]{ref-turi2022tangled}
Turi, A., Rebeleș, M.-R., \& Visu-Petra, L. (2022). The tangled webs
they weave: A scoping review of deception detection and production in
relation to dark triad traits. \emph{Acta Psychologica}, \emph{226},
103574.

\bibitem[\citeproctext]{ref-van2020sex}
Van Den Akker, O. R., Assen, M. A. van, Van Vugt, M., \& Wicherts, J. M.
(2020). Sex differences in trust and trustworthiness: A meta-analysis of
the trust game and the gift-exchange game. \emph{Journal of Economic
Psychology}, \emph{81}, 102329.

\bibitem[\citeproctext]{ref-viola2023designed}
Viola, M., \& Voto, C. (2023). Designed to abuse? Deepfakes and the
non-consensual diffusion of intimate images. \emph{Synthese},
\emph{201}(1), 1--20.

\bibitem[\citeproctext]{ref-wang2024}
Wang, J., \& Nishida, S. (2024). Artificiality is perceptually
associated with trustworthiness but not attractiveness in AI-synthesized
faces. \emph{人工知能学会全国大会論文集 第 38 回 (2024)},
3Xin241--3Xin241.

\bibitem[\citeproctext]{ref-weller2012honest}
Weller, J. A., \& Thulin, E. W. (2012). Do honest people take fewer
risks? Personality correlates of risk-taking to achieve gains and avoid
losses in HEXACO space. \emph{Personality and Individual Differences},
\emph{53}(7), 923--926.

\bibitem[\citeproctext]{ref-wickham2019}
Wickham, H., Averick, M., Bryan, J., Chang, W., McGowan, L. D.,
François, R., Grolemund, G., Hayes, A., Henry, L., Hester, J., Kuhn, M.,
Pedersen, T. L., Miller, E., Bache, S. M., Müller, K., Ooms, J.,
Robinson, D., Seidel, D. P., Spinu, V., \ldots{} Yutani, H. (2019).
Welcome to the {tidyverse}. \emph{Journal of Open Source Software},
\emph{4}(43), 1686. \url{https://doi.org/10.21105/joss.01686}

\bibitem[\citeproctext]{ref-willis2006first}
Willis, J., \& Todorov, A. (2006). First impressions: Making up your
mind after a 100-ms exposure to a face. \emph{Psychological Science},
\emph{17}(7), 592--598.

\end{CSLReferences}






\end{document}
